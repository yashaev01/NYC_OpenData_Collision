
\documentclass{article}
\usepackage{amsmath}

\begin{document}

\section{Hypothesis Testing for Proportions}

\begin{enumerate}
    \item State the Hypotheses:
    \begin{align*}
    H_0 &: P_1 = P_2 \\
    H_a &: P_1 \neq P_2
    \end{align*}
    where \(p_1\) and \(p_2\) are the proportions of injured (or killed) per crash for holidays and non-holidays, respectively.

    \item \(\alpha = 0.05\) is used.

    \item Calculate the Test Statistic:
    The Z-test statistic for the difference between two proportions is given by
    \[
    Z = \frac{{\hat{p}_1 - \hat{p}_2}}{{\sqrt{\hat{p}(1 - \hat{p})\left(\frac{1}{n_1} + \frac{1}{n_2}\right)}}}
    \]
    where
    \begin{itemize}
        \item \(\hat{p}_1\) and \(\hat{p}_2\) are the sample proportions,
        \item \(\hat{p}\) is the pooled sample proportion,
        \item \(n_1\) and \(n_2\) are the sample sizes,
        \[
        \hat{p} = \frac{{n_1 \hat{p}_1 + n_2 \hat{p}_2}}{{n_1 + n_2}}
        \]
    \end{itemize}

    \item Find the P-Value:
    The P-value is the probability of observing a test statistic as extreme as the one calculated, assuming the null hypothesis is true.

    \item Decision Making:
    \begin{itemize}
        \item If \(P < \alpha\), reject the null hypothesis.
        \item If \(P > \alpha\), fail to reject the null hypothesis.
    \end{itemize}
\end{enumerate}

\section{Analysis}
Performingthis test for both the "injured per crash" and "killed per crash" ratios for holidays and non-holidays.
\end{document}

